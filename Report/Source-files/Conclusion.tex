%% LyX 2.3.6.1 created this file.  For more info, see http://www.lyx.org/.
%% Do not edit unless you really know what you are doing.
\documentclass[12pt,english,refpage,intoc,bibliography=totoc,index=totoc,BCOR7.5mm,captions=tableheading]{scrbook}
\usepackage[T1]{fontenc}
\usepackage[latin9]{inputenc}
\setcounter{secnumdepth}{3}
\usepackage{color}
\usepackage{babel}
\usepackage[unicode=true,
 bookmarks=true,bookmarksnumbered=true,bookmarksopen=false,
 breaklinks=false,pdfborder={0 0 1},backref=false,colorlinks=true]
 {hyperref}
\hypersetup{pdftitle={The LyX User's Guide},
 pdfauthor={LyX Team},
 pdfsubject={LyX},
 pdfkeywords={LyX},
 linkcolor=black, citecolor=black, urlcolor=blue, filecolor=blue, pdfpagelayout=OneColumn, pdfnewwindow=true, pdfstartview=XYZ, plainpages=false}

\makeatletter
%%%%%%%%%%%%%%%%%%%%%%%%%%%%%% User specified LaTeX commands.
% DO NOT ALTER THIS PREAMBLE!!!
%
% This preamble is designed to ensure that the User's Guide prints
% out as advertised. If you mess with this preamble,
% parts of the User's Guide may not print out as expected.  If you
% have problems LaTeXing this file, please contact 
% the documentation team
% email: lyx-docs@lists.lyx.org

\usepackage{ifpdf} % part of the hyperref bundle
\ifpdf % if pdflatex is used

 % set fonts for nicer pdf view
 \IfFileExists{lmodern.sty}{\usepackage{lmodern}}{}

\fi % end if pdflatex is used

% for correct jump positions whe clicking on a link to a float
\usepackage[figure]{hypcap}

% the pages of the TOC is numbered roman
% and a pdf-bookmark for the TOC is added
\let\myTOC\tableofcontents
\renewcommand\tableofcontents{%
  \frontmatter
  \pdfbookmark[1]{\contentsname}{}
  \myTOC
  \mainmatter }

% define a short command for \textvisiblespace
\newcommand{\spce}{\textvisiblespace}

% macro for italic page numbers in the index
\newcommand{\IndexDef}[1]{\textit{#1}}

% for customized page headers/footers
% only needed because they are only used in one section of the document
\usepackage{fancyhdr}
% change header rule width
\renewcommand{\headrulewidth}{2pt}

% workaround for a makeindex bug,
% see sec. "Index Entry Order"
% only uncomment this when you are using makindex
%\let\OrgIndex\index 
%\renewcommand*{\index}[1]{\OrgIndex{#1}}

\makeatother

\begin{document}

\section*{Conclusion}

The goal of this project is to study a stochastic approach called
particle filter to solve the problem of failure diagnosis and prognosis.
We chose the Gamma process to model degradation. After understanding
how to construct a particle filter, we apply the algorithms to estimate
the current level of degradation and then calculate the residual lifetime
of the system of interest. Using simulation, we evaluate the performance
of the particle filter when some important parameters are varied.
In addition, the behavior of particle filter when dealing with two
Gamma processes of different increments is also considered. To conclude
the study, we give some important remarks by analyzing the simulation
results.

In the literature, we notice that sophisticated extensions of particle
filter are developed. Also, further work is in progress to improve
the performance of particle filter. For example, it is desirable to
increase significantly the quality of the estimation by using cleverly
the collected data since too many uncertain measurement values can
perturb the particle filter. In fact, the time at which we carry out
the measurements could be periodic or following a certain probability
law, i.e., the time of doing measurements is a random variable. This
also helps us to reduce inspection cost.

Within this project, we verified the algorithms of particle filter
with simulated data. It is, nevertheless, more convincing if we can
have access to real monitoring data acquired from a real degradation
process to examine the efficiency of these algorithms.
\end{document}
