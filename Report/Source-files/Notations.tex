
\section*{Notations}

Pour mieux comprendre le rapport, il faut faire attention � des notations
suivantes:
\begin{itemize}
\item $X_{t}=\left\{ x_{0},\,x_{1},\text{\dots},\,x_{t}\right\} $ repr�sente
l'ensemble des �tats r�els (cach�s) du processus de d�gradation jusqu'�
l'instant $\left(t\right)$.
\item $Y_{t}=\left\{ y_{0},\,y_{1},\text{\dots},\,y_{t}\right\} $ repr�sente
l'ensemble des valeurs de mesures correspondant respectivement � des
�tats $\left(x_{t}\right)$. En supposant que au d�but $\left(t=0\right)$,
on n'effectue pas aucune mesure, donc $Y_{t}=\left\{ y_{1},\text{\dots},\,y_{t}\right\} $
effectivement.
\item $\left(x_{t}^{i},i=1:N_{s}\right)$ d�signe l'�chantillon $\left(i\right)$
de l'�tat $\left(x_{t}\right)$.
\item $\left\{ x_{t}^{i}\right\} _{i=1}^{N_{s}}$ d�signe un ensemble contenant
$N_{s}$ �chantillons.
\item Dans ce rapport, quand on parle d'un �chantillon, on souhaite parler
de la position de cet �chantillon. Et quand cet �chantillon est assign�
d'un poids, il devient une \textit{particule. }
\item Dans quelques articles sur le filtre particulaire, le terme ``\textit{filtering
distribution}'' est utilis� pour d�signer la loi marginale\textit{
$p\left(x_{t}\mid Y_{t}\right)$ }de la loi a posteriori\textit{ }$p\left(X_{t}\mid Y_{t}\right)$.
Pour la raison de simplicit�, dans ce rapport on utilise abusivement
le terme loi \textit{a posteriori} pour indiquer cette loi marginale.
En effet, avant d'estimer la vie r�siduelle � un instant donn� $\left(t\right)$,
il s'agit d'un probl�me d'estimer le niveau de d�gradation $\left(x_{t}\right)$
� cet instant en tenant compte des valeurs de mesures $\left(Y_{t}\right)$.
Ce travail pr�liminaire peut �tre r�alis� par l'�tude de la loi \textit{a
posteriori $p\left(x_{t}\mid Y_{t}\right)$. }
\end{itemize}

