%% LyX 2.3.6.1 created this file.  For more info, see http://www.lyx.org/.
%% Do not edit unless you really know what you are doing.
\documentclass[12pt,english,refpage,intoc,bibliography=totoc,index=totoc,BCOR7.5mm,captions=tableheading]{scrbook}
\usepackage[T1]{fontenc}
\usepackage[latin9]{inputenc}
\setcounter{secnumdepth}{3}
\usepackage{color}
\usepackage{babel}
\usepackage{amstext}
\usepackage[unicode=true,
 bookmarks=true,bookmarksnumbered=true,bookmarksopen=false,
 breaklinks=false,pdfborder={0 0 1},backref=false,colorlinks=true]
 {hyperref}
\hypersetup{pdftitle={The LyX User's Guide},
 pdfauthor={LyX Team},
 pdfsubject={LyX},
 pdfkeywords={LyX},
 linkcolor=black, citecolor=black, urlcolor=blue, filecolor=blue, pdfpagelayout=OneColumn, pdfnewwindow=true, pdfstartview=XYZ, plainpages=false}

\makeatletter
%%%%%%%%%%%%%%%%%%%%%%%%%%%%%% User specified LaTeX commands.
% DO NOT ALTER THIS PREAMBLE!!!
%
% This preamble is designed to ensure that the User's Guide prints
% out as advertised. If you mess with this preamble,
% parts of the User's Guide may not print out as expected.  If you
% have problems LaTeXing this file, please contact 
% the documentation team
% email: lyx-docs@lists.lyx.org

\usepackage{ifpdf} % part of the hyperref bundle
\ifpdf % if pdflatex is used

 % set fonts for nicer pdf view
 \IfFileExists{lmodern.sty}{\usepackage{lmodern}}{}

\fi % end if pdflatex is used

% for correct jump positions whe clicking on a link to a float
\usepackage[figure]{hypcap}

% the pages of the TOC is numbered roman
% and a pdf-bookmark for the TOC is added
\let\myTOC\tableofcontents
\renewcommand\tableofcontents{%
  \frontmatter
  \pdfbookmark[1]{\contentsname}{}
  \myTOC
  \mainmatter }

% define a short command for \textvisiblespace
\newcommand{\spce}{\textvisiblespace}

% macro for italic page numbers in the index
\newcommand{\IndexDef}[1]{\textit{#1}}

% for customized page headers/footers
% only needed because they are only used in one section of the document
\usepackage{fancyhdr}
% change header rule width
\renewcommand{\headrulewidth}{2pt}

% workaround for a makeindex bug,
% see sec. "Index Entry Order"
% only uncomment this when you are using makindex
%\let\OrgIndex\index 
%\renewcommand*{\index}[1]{\OrgIndex{#1}}

\makeatother

\begin{document}

\section*{Problem statement}

The residual life (RUL) represents the remaining operating time of
a component or a system before its failure date. This value is an
important factor to consider when scheduling maintenance operations.
In general, the RUL is estimated using data obtained from deterioration
measurements up to the current moment. However, because of sensors\textquoteright s
imperfection and the impact of the working environment, the monitored
data are often contaminated. This raises difficulty for diagnosis
(retrieving the current level of degradation) and, consequently, prognosis
(predicting the future evolution of the degradation state). To address
such a problem, this study pays particular attention to the use of
the particle filter technique.\\

Using sequential Monte Carlo method, the particle filter is an implementation
of recursive Bayesian estimation. The idea is to estimate at each
time $\left(t\right)$, the \textit{a posteriori} distribution $p\left(x_{t}\mid Y_{t}\right)$
of the degradation level by a discrete distribution formed by a set
of samples with their associated weights, which we call under the
term \textit{particles} $\left\{ x_{t}^{i},\text{W}_{t}^{i}\right\} _{i=1}^{N_{s}}$.
According to the law of large numbers, when the number of samples
is large, they characterize well the \textit{a posteriori} distribution.
Once the \textit{a posteriori} distribution has been obtained, we
can estimate the actual level of degradation $\left(x_{t}\right)$.
Finally, after estimating $\left(x_{t}\right)$, the calculation of
the RUL is performed using simulation. 
\end{document}
