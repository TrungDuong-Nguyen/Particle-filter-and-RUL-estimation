
\chapter{Conclusion}

L'objectif de ce stage durant six mois est d'�tudier une approche
stochastique dite filtre particulaire pour r�soudre le probl�me de
diagnostic et prognostic de d�faillance. Le processus Gamma est choisi
pour mod�liser la d�gradation. Apr�s avoir chercher � comprendre comment
un filtre particulaire est construit, on mette en application des
algorithmes fondamentales afin de r�aliser l'estimation du niveau
de d�gradation actuel et puis calculer la dur�e de vie r�siduelle
du syst�me d'int�r�t. Dans la simulation, on �tudie la performance
du filtre particulaire lors d'une variation des param�tres importants.
De plus, on consid�re aussi le comportement du filtre particulaire
envers deux processus Gamma de l'incr�ments diff�rents. Des remarques
importantes sont retir�es en analysant les r�sultats de simulation. 

Dans le rapport, des extension sophistiqu�es du filtre particulaire
sont �galement relev�es et d'autres travaux sont en cours pour am�liorer
sa performance. Par exemple, il est souhaitable d'augmenter vivement
la qualit� de l'estimation en utilisant habilement les donn�es disponibles
puisque un grand nombre de valeurs de mesures incertaines peuvent
perturber le filtre particulaire. En effet, les mesures peuvent �tre
effectu�es p�riodiquement ou suivant une loi de probabilit� quelconque,
c'est � dire l'instant de faire la mesure est une variable al�atoire.
Cela nous aide � �conomiser la d�pense.

Les algorithmes du filtre particulaire sont v�rifi�s par la simulation
avec des donn�es cr�es artificiellement. Il est plus convaincant s'il
on a des donn�es r�elles acquis d'un vrai processus de d�gradation
pour examiner l'efficience de ces algorithmes.
